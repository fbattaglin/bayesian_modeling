\documentclass[]{article}
\usepackage{lmodern}
\usepackage{amssymb,amsmath}
\usepackage{ifxetex,ifluatex}
\usepackage{fixltx2e} % provides \textsubscript
\ifnum 0\ifxetex 1\fi\ifluatex 1\fi=0 % if pdftex
  \usepackage[T1]{fontenc}
  \usepackage[utf8]{inputenc}
\else % if luatex or xelatex
  \ifxetex
    \usepackage{mathspec}
  \else
    \usepackage{fontspec}
  \fi
  \defaultfontfeatures{Ligatures=TeX,Scale=MatchLowercase}
\fi
% use upquote if available, for straight quotes in verbatim environments
\IfFileExists{upquote.sty}{\usepackage{upquote}}{}
% use microtype if available
\IfFileExists{microtype.sty}{%
\usepackage{microtype}
\UseMicrotypeSet[protrusion]{basicmath} % disable protrusion for tt fonts
}{}
\usepackage[margin=1in]{geometry}
\usepackage{hyperref}
\hypersetup{unicode=true,
            pdftitle={Bayesian Modeling \& Prediction},
            pdfborder={0 0 0},
            breaklinks=true}
\urlstyle{same}  % don't use monospace font for urls
\usepackage{color}
\usepackage{fancyvrb}
\newcommand{\VerbBar}{|}
\newcommand{\VERB}{\Verb[commandchars=\\\{\}]}
\DefineVerbatimEnvironment{Highlighting}{Verbatim}{commandchars=\\\{\}}
% Add ',fontsize=\small' for more characters per line
\usepackage{framed}
\definecolor{shadecolor}{RGB}{248,248,248}
\newenvironment{Shaded}{\begin{snugshade}}{\end{snugshade}}
\newcommand{\KeywordTok}[1]{\textcolor[rgb]{0.13,0.29,0.53}{\textbf{#1}}}
\newcommand{\DataTypeTok}[1]{\textcolor[rgb]{0.13,0.29,0.53}{#1}}
\newcommand{\DecValTok}[1]{\textcolor[rgb]{0.00,0.00,0.81}{#1}}
\newcommand{\BaseNTok}[1]{\textcolor[rgb]{0.00,0.00,0.81}{#1}}
\newcommand{\FloatTok}[1]{\textcolor[rgb]{0.00,0.00,0.81}{#1}}
\newcommand{\ConstantTok}[1]{\textcolor[rgb]{0.00,0.00,0.00}{#1}}
\newcommand{\CharTok}[1]{\textcolor[rgb]{0.31,0.60,0.02}{#1}}
\newcommand{\SpecialCharTok}[1]{\textcolor[rgb]{0.00,0.00,0.00}{#1}}
\newcommand{\StringTok}[1]{\textcolor[rgb]{0.31,0.60,0.02}{#1}}
\newcommand{\VerbatimStringTok}[1]{\textcolor[rgb]{0.31,0.60,0.02}{#1}}
\newcommand{\SpecialStringTok}[1]{\textcolor[rgb]{0.31,0.60,0.02}{#1}}
\newcommand{\ImportTok}[1]{#1}
\newcommand{\CommentTok}[1]{\textcolor[rgb]{0.56,0.35,0.01}{\textit{#1}}}
\newcommand{\DocumentationTok}[1]{\textcolor[rgb]{0.56,0.35,0.01}{\textbf{\textit{#1}}}}
\newcommand{\AnnotationTok}[1]{\textcolor[rgb]{0.56,0.35,0.01}{\textbf{\textit{#1}}}}
\newcommand{\CommentVarTok}[1]{\textcolor[rgb]{0.56,0.35,0.01}{\textbf{\textit{#1}}}}
\newcommand{\OtherTok}[1]{\textcolor[rgb]{0.56,0.35,0.01}{#1}}
\newcommand{\FunctionTok}[1]{\textcolor[rgb]{0.00,0.00,0.00}{#1}}
\newcommand{\VariableTok}[1]{\textcolor[rgb]{0.00,0.00,0.00}{#1}}
\newcommand{\ControlFlowTok}[1]{\textcolor[rgb]{0.13,0.29,0.53}{\textbf{#1}}}
\newcommand{\OperatorTok}[1]{\textcolor[rgb]{0.81,0.36,0.00}{\textbf{#1}}}
\newcommand{\BuiltInTok}[1]{#1}
\newcommand{\ExtensionTok}[1]{#1}
\newcommand{\PreprocessorTok}[1]{\textcolor[rgb]{0.56,0.35,0.01}{\textit{#1}}}
\newcommand{\AttributeTok}[1]{\textcolor[rgb]{0.77,0.63,0.00}{#1}}
\newcommand{\RegionMarkerTok}[1]{#1}
\newcommand{\InformationTok}[1]{\textcolor[rgb]{0.56,0.35,0.01}{\textbf{\textit{#1}}}}
\newcommand{\WarningTok}[1]{\textcolor[rgb]{0.56,0.35,0.01}{\textbf{\textit{#1}}}}
\newcommand{\AlertTok}[1]{\textcolor[rgb]{0.94,0.16,0.16}{#1}}
\newcommand{\ErrorTok}[1]{\textcolor[rgb]{0.64,0.00,0.00}{\textbf{#1}}}
\newcommand{\NormalTok}[1]{#1}
\usepackage{graphicx,grffile}
\makeatletter
\def\maxwidth{\ifdim\Gin@nat@width>\linewidth\linewidth\else\Gin@nat@width\fi}
\def\maxheight{\ifdim\Gin@nat@height>\textheight\textheight\else\Gin@nat@height\fi}
\makeatother
% Scale images if necessary, so that they will not overflow the page
% margins by default, and it is still possible to overwrite the defaults
% using explicit options in \includegraphics[width, height, ...]{}
\setkeys{Gin}{width=\maxwidth,height=\maxheight,keepaspectratio}
\IfFileExists{parskip.sty}{%
\usepackage{parskip}
}{% else
\setlength{\parindent}{0pt}
\setlength{\parskip}{6pt plus 2pt minus 1pt}
}
\setlength{\emergencystretch}{3em}  % prevent overfull lines
\providecommand{\tightlist}{%
  \setlength{\itemsep}{0pt}\setlength{\parskip}{0pt}}
\setcounter{secnumdepth}{0}
% Redefines (sub)paragraphs to behave more like sections
\ifx\paragraph\undefined\else
\let\oldparagraph\paragraph
\renewcommand{\paragraph}[1]{\oldparagraph{#1}\mbox{}}
\fi
\ifx\subparagraph\undefined\else
\let\oldsubparagraph\subparagraph
\renewcommand{\subparagraph}[1]{\oldsubparagraph{#1}\mbox{}}
\fi

%%% Use protect on footnotes to avoid problems with footnotes in titles
\let\rmarkdownfootnote\footnote%
\def\footnote{\protect\rmarkdownfootnote}

%%% Change title format to be more compact
\usepackage{titling}

% Create subtitle command for use in maketitle
\newcommand{\subtitle}[1]{
  \posttitle{
    \begin{center}\large#1\end{center}
    }
}

\setlength{\droptitle}{-2em}

  \title{Bayesian Modeling \& Prediction}
    \pretitle{\vspace{\droptitle}\centering\huge}
  \posttitle{\par}
    \author{}
    \preauthor{}\postauthor{}
    \date{}
    \predate{}\postdate{}
  

\begin{document}
\maketitle

\subsection{Setup}\label{setup}

\subsubsection{Load packages}\label{load-packages}

\begin{Shaded}
\begin{Highlighting}[]
\KeywordTok{library}\NormalTok{(ggplot2)}
\KeywordTok{library}\NormalTok{(dplyr)}
\KeywordTok{library}\NormalTok{(statsr)}
\KeywordTok{library}\NormalTok{(BAS)}
\KeywordTok{library}\NormalTok{(reshape2)}
\end{Highlighting}
\end{Shaded}

\subsubsection{Load data}\label{load-data}

Make sure your data and R Markdown files are in the same directory. When
loaded your data file will be called \texttt{movies}. Delete this note
when before you submit your work.

\begin{Shaded}
\begin{Highlighting}[]
\KeywordTok{load}\NormalTok{(}\StringTok{"movies.Rdata"}\NormalTok{)}
\end{Highlighting}
\end{Shaded}

\begin{center}\rule{0.5\linewidth}{\linethickness}\end{center}

\subsection{Part 1: Data}\label{part-1-data}

The dataset contains information about movies in Rotten Tomatoes and
IMDB. Both are considered review-aggregation websites for films. On
IMDb, all films are given an overall rating out of ten. In a roundabout
way, these ratings are derived from votes submitted by IMDb users, not
movie critics. Rotten Tomatoes gives films a score out of 100 based on
the averaged reviews of professional film critics.

\subsubsection{Metodology and Sampling}\label{metodology-and-sampling}

There are 651 randomly sampled movies produced and released before 2016.
There are 32 available variables. Note that the sample size is very
small, compared with the population of movie watchers as a whole, which
also does not allow us to draw conclusions. This is an observational
study therefore we cannot establish causality.

\subparagraph{Some Possible Sources of
Bias}\label{some-possible-sources-of-bias}

\begin{itemize}
\tightlist
\item
  The data is collected from movie fans, therefore may not represent the
  average movie goer. It is possibe these can be sources of
  \textbf{sampling bias}.
\item
  The study sampling is random, however, so the results are
  generalizable to movies produced and released before 2016 in the US.
  However, it will not be generalizable to all movies released in
  \textbf{all parts of the world}.
\end{itemize}

\begin{center}\rule{0.5\linewidth}{\linethickness}\end{center}

\subsection{Part 2: Data manipulation}\label{part-2-data-manipulation}

\begin{enumerate}
\def\labelenumi{\arabic{enumi})}
\item
  Create new variable based on title\_type: New variable should be
  called feature\_film with levels yes (movies that are feature films)
  and no
\item
  Create new variable based on genre: New variable should be called
  drama with levels yes (movies that are dramas) and no
\item
  Create new variable based on mpaa\_rating: New variable should be
  called mpaa\_rating\_R with levels yes (movies that are R rated) and
  no
\item
  Create two new variables based on thtr\_rel\_month:
\end{enumerate}

\begin{itemize}
\tightlist
\item
  New variable called oscar\_season with levels yes (if movie is
  released in November, October, or December) and no
\item
  New variable called summer\_season with levels yes (if movie is
  released in May, June, July, or August) and no
\end{itemize}

\begin{Shaded}
\begin{Highlighting}[]
\NormalTok{movies <-}\StringTok{ }\NormalTok{movies }\OperatorTok
\StringTok{        }\KeywordTok{mutate}\NormalTok{(}\DataTypeTok{feature_film =} \KeywordTok{ifelse}\NormalTok{(title_type }\OperatorTok{==}\StringTok{ "Feature Film"}\NormalTok{,}\StringTok{"yes"}\NormalTok{,}\StringTok{"no"}\NormalTok{),}
               \DataTypeTok{drama =} \KeywordTok{ifelse}\NormalTok{(genre }\OperatorTok{==}\StringTok{ "Drama"}\NormalTok{, }\StringTok{"yes"}\NormalTok{,}\StringTok{"no"}\NormalTok{),}
               \DataTypeTok{mpaa_rating_R =} \KeywordTok{ifelse}\NormalTok{(mpaa_rating }\OperatorTok{==}\StringTok{ "R"}\NormalTok{,}\StringTok{"yes"}\NormalTok{,}\StringTok{"no"}\NormalTok{),}
               \DataTypeTok{oscar_season =} \KeywordTok{ifelse}\NormalTok{(thtr_rel_month }\OperatorTok\StringTok{ }\KeywordTok{c}\NormalTok{(}\DecValTok{10}\NormalTok{,}\DecValTok{11}\NormalTok{, }\DecValTok{12}\NormalTok{),}\StringTok{"yes"}\NormalTok{,}\StringTok{"no"}\NormalTok{),}
               \DataTypeTok{summer_season =} \KeywordTok{ifelse}\NormalTok{(thtr_rel_month }\OperatorTok\StringTok{ }\KeywordTok{c}\NormalTok{(}\DecValTok{5}\NormalTok{, }\DecValTok{6}\NormalTok{, }\DecValTok{7}\NormalTok{, }\DecValTok{8}\NormalTok{),}\StringTok{"yes"}\NormalTok{,}\StringTok{"no"}\NormalTok{))}
\end{Highlighting}
\end{Shaded}

\begin{center}\rule{0.5\linewidth}{\linethickness}\end{center}

\subsection{Part 3: Exploratory data
analysis}\label{part-3-exploratory-data-analysis}

First we need to create a new dataframe using only the new variables
that we created in the exercise above. Next, we will ``melt'' this
dataframe in a way that is most easily readable by the summary and plot
functions.

\begin{Shaded}
\begin{Highlighting}[]
\NormalTok{newVars <-}\StringTok{ }\KeywordTok{select}\NormalTok{(movies, audience_score, feature_film, drama, mpaa_rating_R, oscar_season, summer_season)}
\NormalTok{dfmelt <-}\StringTok{ }\KeywordTok{melt}\NormalTok{(newVars, }\DataTypeTok{id.vars=}\DecValTok{1}\NormalTok{)}

\NormalTok{dfmelt }\OperatorTok
\StringTok{        }\KeywordTok{group_by}\NormalTok{(variable,value) }\OperatorTok
\StringTok{        }\KeywordTok{summarize}\NormalTok{(}\DataTypeTok{avg_rating =} \KeywordTok{mean}\NormalTok{(audience_score), }
                  \DataTypeTok{min_rating =} \KeywordTok{min}\NormalTok{(audience_score), }
                  \DataTypeTok{max_rating =} \KeywordTok{max}\NormalTok{(audience_score))}
\end{Highlighting}
\end{Shaded}

\begin{verbatim}
## # A tibble: 10 x 5
## # Groups:   variable [?]
##    variable      value avg_rating min_rating max_rating
##    <fct>         <chr>      <dbl>      <dbl>      <dbl>
##  1 feature_film  no          81.0         19         96
##  2 feature_film  yes         60.5         11         97
##  3 drama         no          59.7         11         97
##  4 drama         yes         65.3         13         95
##  5 mpaa_rating_R no          62.7         11         96
##  6 mpaa_rating_R yes         62.0         14         97
##  7 oscar_season  no          61.8         11         96
##  8 oscar_season  yes         63.7         13         97
##  9 summer_season no          62.6         13         97
## 10 summer_season yes         61.8         11         94
\end{verbatim}

The most interesting variable appears to be feature\_film. The average
for non-feature films is 81.05, while the average for feature films is
60.46. We'll see if this ends up being an important factor in the model.

\begin{Shaded}
\begin{Highlighting}[]
\KeywordTok{ggplot}\NormalTok{(dfmelt, }\KeywordTok{aes}\NormalTok{(}\DataTypeTok{x=}\NormalTok{value, }\DataTypeTok{y=}\NormalTok{audience_score,}\DataTypeTok{fill=}\NormalTok{variable))}\OperatorTok{+}
\StringTok{  }\KeywordTok{geom_boxplot}\NormalTok{(}\DataTypeTok{alpha=}\FloatTok{0.4}\NormalTok{)}\OperatorTok{+}
\StringTok{  }\KeywordTok{facet_grid}\NormalTok{(.}\OperatorTok{~}\NormalTok{variable)}\OperatorTok{+}
\StringTok{  }\KeywordTok{scale_fill_brewer}\NormalTok{(}\DataTypeTok{palette=}\StringTok{"Dark2"}\NormalTok{)}\OperatorTok{+}
\StringTok{  }\KeywordTok{labs}\NormalTok{(}\DataTypeTok{x=}\StringTok{"New Variables"}\NormalTok{,}\DataTypeTok{y=}\StringTok{"Audience Score"}\NormalTok{)}
\end{Highlighting}
\end{Shaded}

\includegraphics{bayesian_project_files/figure-latex/unnamed-chunk-3-1.pdf}

By looking at the boxplot, we can see that except for feature film,
other variables have little effect on the audience\_score. Maybe that
could be explained by the fact that documentaries tend to attract very
especific types of movie-goers. This audience tend to appreciate the
informative aspect of the genre.

One more interesting statistic is that the genre drama's median score
seems to be higher than other genres.

\begin{center}\rule{0.5\linewidth}{\linethickness}\end{center}

\subsection{Part 4: Modeling}\label{part-4-modeling}

First we will start using the full model with the selected variables for
the exercise: feature\_film, drama, runtime, mpaa\_rating\_R,
thtr\_rel\_year, oscar\_season, summer\_season, imdb\_rating,
imdb\_num\_votes, critics\_score, best\_pic\_nom, best\_pic\_win,
best\_actor\_win, best\_actress\_win, best\_dir\_win, top200\_box.

\begin{Shaded}
\begin{Highlighting}[]
\KeywordTok{set.seed}\NormalTok{(}\DecValTok{1234}\NormalTok{)}

\NormalTok{df<-movies }\OperatorTok
\StringTok{  }\KeywordTok{select}\NormalTok{(feature_film,drama,runtime,mpaa_rating_R,}
\NormalTok{                     thtr_rel_year,oscar_season,summer_season,imdb_rating,}
\NormalTok{                     imdb_num_votes,critics_score,best_pic_nom,}
\NormalTok{                     best_pic_win,best_actor_win,best_actress_win,}
\NormalTok{                     best_dir_win,top200_box,audience_score)}
\end{Highlighting}
\end{Shaded}

Often, several models are equally plausible and choosing only one
ignores the inherent uncertainty involved in choosing the variables to
include in the model. A way to get around this problem is to implement
Bayesian model averaging (BMA), in which multiple models are averaged to
obtain posteriors of coefficients and predictions from new data.

\begin{Shaded}
\begin{Highlighting}[]
\NormalTok{model1 <-}\StringTok{ }\KeywordTok{bas.lm}\NormalTok{(audience_score }\OperatorTok{~}\StringTok{ }\NormalTok{., }\DataTypeTok{data=}\NormalTok{df,}
               \DataTypeTok{prior=}\StringTok{"BIC"}\NormalTok{,}
               \DataTypeTok{modelprior =} \KeywordTok{uniform}\NormalTok{())}
\end{Highlighting}
\end{Shaded}

\begin{verbatim}
## Warning in bas.lm(audience_score ~ ., data = df, prior = "BIC", modelprior
## = uniform()): dropping 1 rows due to missing data
\end{verbatim}

\begin{Shaded}
\begin{Highlighting}[]
\KeywordTok{summary}\NormalTok{(model1)}
\end{Highlighting}
\end{Shaded}

\begin{verbatim}
##                     P(B != 0 | Y)    model 1       model 2       model 3
## Intercept              1.00000000     1.0000     1.0000000     1.0000000
## feature_filmyes        0.06536947     0.0000     0.0000000     0.0000000
## dramayes               0.04319833     0.0000     0.0000000     0.0000000
## runtime                0.46971477     1.0000     0.0000000     0.0000000
## mpaa_rating_Ryes       0.19984016     0.0000     0.0000000     0.0000000
## thtr_rel_year          0.09068970     0.0000     0.0000000     0.0000000
## oscar_seasonyes        0.07505684     0.0000     0.0000000     0.0000000
## summer_seasonyes       0.08042023     0.0000     0.0000000     0.0000000
## imdb_rating            1.00000000     1.0000     1.0000000     1.0000000
## imdb_num_votes         0.05773502     0.0000     0.0000000     0.0000000
## critics_score          0.88855056     1.0000     1.0000000     1.0000000
## best_pic_nomyes        0.13119140     0.0000     0.0000000     0.0000000
## best_pic_winyes        0.03984766     0.0000     0.0000000     0.0000000
## best_actor_winyes      0.14434896     0.0000     0.0000000     1.0000000
## best_actress_winyes    0.14128087     0.0000     0.0000000     0.0000000
## best_dir_winyes        0.06693898     0.0000     0.0000000     0.0000000
## top200_boxyes          0.04762234     0.0000     0.0000000     0.0000000
## BF                             NA     1.0000     0.9968489     0.2543185
## PostProbs                      NA     0.1297     0.1293000     0.0330000
## R2                             NA     0.7549     0.7525000     0.7539000
## dim                            NA     4.0000     3.0000000     4.0000000
## logmarg                        NA -3615.2791 -3615.2822108 -3616.6482224
##                           model 4       model 5
## Intercept               1.0000000     1.0000000
## feature_filmyes         0.0000000     0.0000000
## dramayes                0.0000000     0.0000000
## runtime                 0.0000000     1.0000000
## mpaa_rating_Ryes        1.0000000     1.0000000
## thtr_rel_year           0.0000000     0.0000000
## oscar_seasonyes         0.0000000     0.0000000
## summer_seasonyes        0.0000000     0.0000000
## imdb_rating             1.0000000     1.0000000
## imdb_num_votes          0.0000000     0.0000000
## critics_score           1.0000000     1.0000000
## best_pic_nomyes         0.0000000     0.0000000
## best_pic_winyes         0.0000000     0.0000000
## best_actor_winyes       0.0000000     0.0000000
## best_actress_winyes     0.0000000     0.0000000
## best_dir_winyes         0.0000000     0.0000000
## top200_boxyes           0.0000000     0.0000000
## BF                      0.2521327     0.2391994
## PostProbs               0.0327000     0.0310000
## R2                      0.7539000     0.7563000
## dim                     4.0000000     5.0000000
## logmarg             -3616.6568544 -3616.7095127
\end{verbatim}

\begin{Shaded}
\begin{Highlighting}[]
\KeywordTok{image}\NormalTok{(model1,}\DataTypeTok{rotate=}\OtherTok{FALSE}\NormalTok{)}
\end{Highlighting}
\end{Shaded}

\includegraphics{bayesian_project_files/figure-latex/unnamed-chunk-5-1.pdf}

As we can see in the summary the most effcient model, which has
posterior probability of 0.1297, includes only Intecept, runtime,
imdb\_rating, and critics\_score. Therefore we will try a new model
using only these variables.

\begin{Shaded}
\begin{Highlighting}[]
\NormalTok{df2 <-}\StringTok{ }\NormalTok{movies }\OperatorTok
\StringTok{  }\KeywordTok{select}\NormalTok{(runtime,imdb_rating,critics_score, audience_score)}

\NormalTok{model2 <-}\StringTok{ }\KeywordTok{bas.lm}\NormalTok{(audience_score }\OperatorTok{~}\StringTok{ }\NormalTok{., }\DataTypeTok{data=}\NormalTok{df2,}
               \DataTypeTok{prior=}\StringTok{"BIC"}\NormalTok{,}
               \DataTypeTok{modelprior =} \KeywordTok{uniform}\NormalTok{())}
\end{Highlighting}
\end{Shaded}

\begin{verbatim}
## Warning in bas.lm(audience_score ~ ., data = df2, prior = "BIC", modelprior
## = uniform()): dropping 1 rows due to missing data
\end{verbatim}

\begin{Shaded}
\begin{Highlighting}[]
\NormalTok{model2}
\end{Highlighting}
\end{Shaded}

\begin{verbatim}
## 
## Call:
## bas.lm(formula = audience_score ~ ., data = df2, prior = "BIC", 
##     modelprior = uniform())
## 
## 
##  Marginal Posterior Inclusion Probabilities: 
##     Intercept        runtime    imdb_rating  critics_score  
##        1.0000         0.5102         1.0000         0.9051
\end{verbatim}

We want to verify our final model using the plot function to generate
different visualizations.

\begin{Shaded}
\begin{Highlighting}[]
\KeywordTok{plot}\NormalTok{(model2)}
\end{Highlighting}
\end{Shaded}

\includegraphics{bayesian_project_files/figure-latex/unnamed-chunk-7-1.pdf}
\includegraphics{bayesian_project_files/figure-latex/unnamed-chunk-7-2.pdf}
\includegraphics{bayesian_project_files/figure-latex/unnamed-chunk-7-3.pdf}
\includegraphics{bayesian_project_files/figure-latex/unnamed-chunk-7-4.pdf}

\begin{Shaded}
\begin{Highlighting}[]
\KeywordTok{confint}\NormalTok{(}\KeywordTok{coefficients}\NormalTok{(model2))}
\end{Highlighting}
\end{Shaded}

\begin{verbatim}
##                      2.5%      97.5%        beta
## Intercept     61.59918898 63.1369749 62.34769231
## runtime       -0.08195701  0.0000000 -0.02755783
## imdb_rating   13.62963998 16.4862308 14.96315740
## critics_score  0.00000000  0.1061856  0.06498060
## attr(,"Probability")
## [1] 0.95
## attr(,"class")
## [1] "confint.bas"
\end{verbatim}

Based on the credicle interval above we can draw the following model to
predict the audience score:

\[Audience Score = 62.347 - 0.0276*runtime + 14.963*imdbRating + 0.0649*criticsScore\]

\begin{center}\rule{0.5\linewidth}{\linethickness}\end{center}

\subsection{Part 5: Prediction}\label{part-5-prediction}

Now we will use our model to predict the audience\_score of the movie
The Arrival (2016).

First let us find the predictive values under the Bayesian Model
Averaging. Then we will use the \emph{Best Predictive Model}
(\texttt{BPM}), the one which has predictions closest to BMA and
corresponding posterior standard deviations. And finally the
\emph{Median Probability Model} (\texttt{MPM}) .

\begin{Shaded}
\begin{Highlighting}[]
\CommentTok{#arrival data}
\NormalTok{arrival <-}\StringTok{ }\KeywordTok{data.frame}\NormalTok{(}\DataTypeTok{runtime=}\DecValTok{116}\NormalTok{, }\DataTypeTok{imdb_rating =} \FloatTok{7.9}\NormalTok{, }\DataTypeTok{critics_score =} \DecValTok{84}\NormalTok{)}

\NormalTok{predict1 <-}\StringTok{ }\KeywordTok{predict}\NormalTok{(model2,arrival,}\DataTypeTok{estimator=}\StringTok{"BMA"}\NormalTok{, }\DataTypeTok{se.fit=}\OtherTok{TRUE}\NormalTok{)}
\NormalTok{predict2 <-}\StringTok{ }\KeywordTok{predict}\NormalTok{(model2,arrival,}\DataTypeTok{estimator=}\StringTok{"BPM"}\NormalTok{, }\DataTypeTok{se.fit=}\OtherTok{TRUE}\NormalTok{)}
\NormalTok{predict3 <-}\StringTok{ }\KeywordTok{predict}\NormalTok{(model2,arrival,}\DataTypeTok{estimator=}\StringTok{"MPM"}\NormalTok{, }\DataTypeTok{se.fit=}\OtherTok{TRUE}\NormalTok{)}
\end{Highlighting}
\end{Shaded}

\begin{verbatim}
## Warning in bas.lm(eval(object$call$formula), data =
## eval(object$call$data, : dropping 1 rows due to missing data
\end{verbatim}

\begin{Shaded}
\begin{Highlighting}[]
\NormalTok{ci_bma_movies =}\StringTok{ }\KeywordTok{confint}\NormalTok{(predict1, }\DataTypeTok{estimator=}\StringTok{"BMA"}\NormalTok{)}
\NormalTok{ci_bma_movies}
\end{Highlighting}
\end{Shaded}

\begin{verbatim}
##          2.5%    97.5%     pred
## [1,] 64.70999 104.1935 84.85422
## attr(,"Probability")
## [1] 0.95
## attr(,"class")
## [1] "confint.bas"
\end{verbatim}

\begin{Shaded}
\begin{Highlighting}[]
\NormalTok{ci_bpm_movies =}\StringTok{ }\KeywordTok{confint}\NormalTok{(predict2, }\DataTypeTok{estimator=}\StringTok{"BPM"}\NormalTok{)}
\NormalTok{ci_bpm_movies}
\end{Highlighting}
\end{Shaded}

\begin{verbatim}
##         2.5%    97.5%     pred
## [1,] 64.9701 104.7784 84.87423
## attr(,"Probability")
## [1] 0.95
## attr(,"class")
## [1] "confint.bas"
\end{verbatim}

\begin{Shaded}
\begin{Highlighting}[]
\NormalTok{ci_mpm_movies =}\StringTok{ }\KeywordTok{confint}\NormalTok{(predict3, }\DataTypeTok{estimator=}\StringTok{"MPM"}\NormalTok{)}
\NormalTok{ci_mpm_movies}
\end{Highlighting}
\end{Shaded}

\begin{verbatim}
##          2.5%    97.5%     pred
## [1,] 64.99805 104.5128 84.75543
## attr(,"Probability")
## [1] 0.95
## attr(,"class")
## [1] "confint.bas"
\end{verbatim}

Our model predict the audience score of roughly 84, which is pretty good
compared to the real audience score of 82.

\url{https://www.rottentomatoes.com/m/arrival_2016/}

\begin{center}\rule{0.5\linewidth}{\linethickness}\end{center}

\subsection{Part 6: Conclusion}\label{part-6-conclusion}

Based on the given data, the optimal model we could find contain the
following variables: critics score, runtime and imdb\_rating. These
variables will have the most effects on audience score of a movies on
average. We choose our final model based on the highest posterior
probability.

The model performs well when using the limited variables required by the
assignment. More testing is needed before determining whether this model
is a good fit overall, or just performed well with the single
prediction, as the assignment stipulates.

One shortcoming of this analysis is the lack of prior model. It will
greatly enhance the study if we use a better prior model for our
regression.


\end{document}
